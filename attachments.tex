\chapter{Planning Bridge deepening}
\label{cha:deepening}

\newsavebox\poseserver

\noindent\begin{lrbox}{\poseserver}
  \noindent\begin{lstlisting}[language=C, style=CStyleTextWidth]
while (rclcpp::ok() && !listener_.buffer_.canTransform("map", "base_link", tf2::TimePoint())) rate.sleep();

geometry_msgs::msg::TransformStamped transform_ = listener_.buffer_.lookupTransform("map",  "base_link", tf2::TimePoint());

response->x = transform_.transform.translation.x;
response->y = transform_.transform.translation.y;
  \end{lstlisting}
\end{lrbox}

\newsavebox\navigationclient

\noindent\begin{lrbox}{\navigationclient}
  \noindent\begin{lstlisting}[language=C, style=CStyleNoFrame]
goal_->pose.position.x = request->x;
goal_->pose.position.y = request->y;
goal_->pose.position.z = 0;
goal_->pose.orientation.x = 0.0;
goal_->pose.orientation.y = 0.0;
goal_->pose.orientation.z = 0.0;
goal_->pose.orientation.w = 1.0;
goal_->header.frame_id = "map";
goal_->header.stamp = now();
send_goal();
response->success = true;
  \end{lstlisting}
\end{lrbox}

\noindent\begin{figure}[h]
  \centering
    \noindent\begin{tikzpicture}
      \node [poseserver] (pose-server) {\usebox\poseserver};
      \node[fill=white] at (pose-server.south) {POSE SERVER};
      
      \node [inception, above left=5.7cm and -4.625cm of pose-server] (do-work) {DO WORK};
      \node [inception, above left=3.8cm and -4.625cm of pose-server] (get-pose) {GET POSE};
      \node [inception, above left=1.9cm and -4.625cm of pose-server] (continue-work) {CONTINUE WORK};
      \node[inception, inner xsep=1em, inner ysep=1em, fit=(do-work) (continue-work), above left=1.5cm and -5cm of pose-server] (plan-clients) {};
      % \node [poseserver, above=1cm of pose-server, fit=(do-work) (continue-work)] (plan-clients) {PLANNING CLIENTS};  
      \node[fill=white] at (plan-clients.south) {PLANNING CLIENTS};

      \node [navigationclient, right=5cm of plan-clients] (nav-client) {\usebox\navigationclient};
      \node[fill=white] at (nav-client.south) {NAVIGATION CLIENT};

      \node [block, above=3cm of plan-clients] (plan-server) {PLANNING SERVER};
      \node [block, right=7.9cm of plan-server] (nav-server) {NAVIGATION SERVER};

      \draw [arrow] (plan-server) -- node [left] {plansys2\_msgs} node [right] {::msg::ActionExecution} (plan-clients);
      \draw [arrow] (nav-client) -- node [left] {nav2\_msgs::action} node [right] {::NavigateToPose} (nav-server);
      \draw [arrow] (do-work.7) -- node [above] {float32 x, y} (nav-client.148);
      \draw [arrow] (nav-client.154) -- node [below] {bool success} (do-work.353);
      \draw [arrow] (get-pose.353) -| (pose-server.95);
      \draw [arrow] (pose-server.85) |- node [above left] {float32 x, y} (get-pose.7);

      \draw [arrow] (do-work) -- (get-pose);
      \draw [arrow] (get-pose) -- (continue-work);

  \end{tikzpicture}
  \caption[Nodes interaction deepening]{\textbf{DO WORK}, \textbf{GET POSE} and \textbf{CONTINUE WORK} functions are shown in the next page.}
\end{figure}

\newpage

\subsection{DO WORK function}

\noindent\begin{lstlisting}[language=C, style=CStyle, caption={Starts navigation with the given goal coordinates}]
if(!is_initial_distance_set_){
  room_goal_ = current_arguments_[GOAL_INDEX];
  goal_position_ = labels_.at(room_goal_).at("center").get<std::vector<double>>();
  
  auto request = std::make_shared<planning_bridge_msgs::srv::StartNavigation::Request>();
  request->x = goal_position_[0];
  request->y = goal_position_[1];
  
  start_navigation_client_->async_send_request(request,
  [this](rclcpp::Client<planning_bridge_msgs::srv::StartNavigation>::SharedFuture future){
    auto result = future.get();
    if(result->success) {
      RCLCPP_INFO(get_logger(), "Navigation started");
      get_pose();
    } else {
      RCLCPP_ERROR(get_logger(), "Navigation failed");
    }
  });
  
} else {
  get_pose();
}
\end{lstlisting}

\subsection{GET POSE function}

\noindent\begin{lstlisting}[language=C, style=CStyle, caption={Asks pose and sets initial and current poses}]
current_pose_client_ptr_->async_send_request(
  std::make_unique<planning_bridge_msgs::srv::CurrentPose::Request>(),
  [this](rclcpp::Client<planning_bridge_msgs::srv::CurrentPose>::SharedFuture future) {
    auto result = future.get();
    if(!is_initial_distance_set_){
      initial_distance_ = std::sqrt(std::pow(result->x - goal_position_[0], 2) +
      std::pow(result->y - goal_position_[1], 2));
      is_initial_distance_set_ = true;
      current_distance_ = initial_distance_;
    } else {
      current_distance_ = std::sqrt(std::pow(result->x - goal_position_[0], 2) +
      std::pow(result->y - goal_position_[1], 2));
      if(current_distance_ > initial_distance_){
        float temp = current_distance_;
        current_distance_ = initial_distance_;
        initial_distance_ = temp;
      }
    }
    continue_work();
    });
\end{lstlisting}

\newpage

\subsection{CONTINUE WORK function}

\noindent\begin{lstlisting}[language=C, style=CStyle, caption={Sends feedback until the goal is reached}]
progress_ = 1.0 - (current_distance_ /
initial_distance_);

if (progress_ < 0.95) {
  send_feedback(progress_, "Move running");
} else {
  finish(true, 1.0, "Move completed");

  progress_ = 0.0;
  is_initial_distance_set_ = false;
  std::cout << std::endl;
}

std::cout << "\r\e[K" << std::flush;
std::cout << "Moving ... [" << std::min(100.0, progress_ * 100.0) << "%]" << std::flush;
  \end{lstlisting}

\chapter{How to}

As described in \autoref{cha:techstack}, \code{docker compose} has been used in order to easily \textbf{deploy} the system. However, \textbf{several configurations} are available (e.g. run everything on the robot, run only robot nodes on the robot, run \acrshort{slam}, run simulation), and you should edit the \code{docker-compose.yaml} file each time to meet your needs, or have multiple files containing different configurations. To simplify it, some Python scripts have been created:

\begin{itemize}
  \item \code{start.py}
  \item \code{attach.py}
  \item \code{stop.py}
\end{itemize}

\subsection{start.py}

\begin{lstlisting}[language=bash, style=bashStyle, caption={Generates a \code{docker-compose.yaml} file from a template and runs it}]
usage: start.py [-h] [-r] [-g] [-n] [-s] [-m] [-p] [-d]

start desired nodes containers

optional arguments:
  -h, --help         show this help message and exit
  -r, --robot        start robot nodes
  -g, --gazebo       start simulation nodes
  -n, --navigation  start navigation nodes
  -s, --slam         start slam nodes
  -m, --map          save slam map
  -p, --planning     start planning nodes
  -d, --detached     run containers in detached mode
\end{lstlisting}

\noindent You can decide \textbf{which nodes} you want to run by passing the corresponding arguments\footnote{Some of them may be mutually exclusive}. Note that each time, before running the actual nodes, the workspace will be \textbf{recompiled}, to include any latest changes.

\subsection{attach.py}

\begin{lstlisting}[language=bash, style=bashStyle]
usage: attach.py [-h] [-r | -b | -g | -n | -s | -m | -p | -u | -w]

attach to a desired container

optional arguments:
  -h, --help         show this help message and exit
  -r, --robot        attach to robot container
  -b, --bridge       attach to ROS1-ROS2 bridge node
  -g, --gazebo       attach to simulation container
  -n, --navigation  attach to navigation container
  -s, --slam         attach to slam container
  -m, --map          attach to map container
  -p, --planning     attach to planning container
  -u, --ugv          attach to UGV actions container
  -w, --webserver    attach to webserver container  
\end{lstlisting} 

\noindent It is possible to attach to only one container at a time, since each argument is \textbf{mutually exclusive}. This is useful to \textbf{debug} the system, or to \textbf{run commands} inside the container.

\subsection{stop.py}

\begin{lstlisting}[language=bash, style=bashStyle]
usage: remove.py [-h]

remove running or stopped containers

optional arguments:
  -h, --help  show this help message and exit
\end{lstlisting}