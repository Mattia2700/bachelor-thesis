\chapter{Introduction} % senza numerazione
\label{cha:intro}

\section{Summary}

The goal of this project, as described in the abstract, aims to develop a complete system for \textbf{surveillance proposes} using ground robots, but also flying ones.
Since the project is quite complex, it is \textbf{shared} among other two students, with each one of us focusing into a specific subtask. It is constituted of: 

\begin{itemize}
  \item a \textbf{planning system}, responsible for dispatching the surveillance tasks to \acrfull{ugvs} and/or \acrfull{uavs} the best possible way, based on their battery charge, giving them some waypoints to reach 
  \item a solution to autonomously control the \textbf{drones} (\acrshort{uavs})
  \item a solution to autonomously control ground robots avoiding \textbf{dynamic obstacles} (\acrshort{ugvs})
\end{itemize}

Everything is developed using \acrshort{ros}2, that gives us the possibility to work \textbf{separately} on your own project, and thanks to \acrshort{ros} \textbf{modularity}, at the end we only need to put everything together.

The reason behind my choice about the internship and this consequent thesis is leaded by my \textbf{growing interest} in this topic: just some month before taking part in this work, I attended a course about \textit{robotic fundamentals}\cite{intro2robotics} where we also developed a project quite similar to this, but with a robotic manipulator.

For me, is the possibility to give to some inanimate object, like a wheeled robot or a manipulator, something that could be described as \textbf{intelligence}, the ability to perform some tasks in response of other ones, figuring out which ones are the \textbf{best suitable} for each particular situation. Some challenges I decided to took are: the choice of the \textbf{new version} of ROS, with some great improves respect to the older one, but with less community support, and also working for the first time on a \textbf{mobile robot}.

\section{Other projects involved}

What follows is a brief description of the work done by the other two students to have a better and clear idea of the workflow of the entire system.

\subsection{Planning system, fleet management and web interface}
\label{sub:planning}

It is brain of the system, it lets the user choose which type of robot he wants to use (only ground robots or drones with ground robots as auxiliary), and then it dispatches the tasks to the right robots. It is possible to define multiple goals for each robot, and once done, the planner will start figuring out the best way to organize the fleet.

In order to resolve a problem it must have been defined a \textbf{domain}: this contains a description of space you are interested in, in which you could specify \textbf{reachable targets}, making use of a set of \textbf{predefined actions}, that requires some \textbf{preconditions} to be met before executing and causes some \textbf{effects} to the environment. In order to solve a problem, we need to check if its domain is \textbf{compatible} with the planning one, or in other words, if the goal is achievable, and if so, it is possible to look for a \textbf{solution} (i.e., a sequence of actions to be performed).

Each robot waits for new commands to be received, with one specific implementation for each type of task. Here follows a list of actions currently available:

\bigskip

\begin{minipage}[h]{0.45\textwidth}
  \centering
  \textbf{\acrshort{uavs} actions}
  \begin{itemize}
    \centering
    \item \code{uav\_move}
    \item \code{uav\_take\_photo}    
    \item \code{uav\_land\_on\_ugv}
    \item \code{uav\_take\_off\_ugv}
  \end{itemize}
\end{minipage}
\begin{minipage}[h]{0.45\textwidth}
  \centering
  \textbf{\acrshort{ugvs} actions}
    \begin{itemize}
      \centering
      \item \code{ugv\_move}
      \item \code{ugv\_charge}
      \item \code{try\_ugv\_charge}
      \item \code{ugv\_transporting\_uav\_move}
    \end{itemize}
\end{minipage}

\bigskip

These actions are the only ones the planner uses when a problem is specified to find a possible outcome: the output are \textbf{waterfall actions} that should be done, so when one is completed it is possible to move to the next one. The implementation is left to whom works on that robot, because of their \textbf{better understanding} of that particular robot.

But, speaking about movement, we need some \textbf{coordinates}. These are extracted from the 3D mesh and thanks to an \textbf{annotation program} it is possible to assign them a name: in this way instead of using a bunch of numbers, we use the name of the room, and their relationship is defined in a file. Each room is then \textbf{connected} to the others by default.

Thanks to a web interface, communicating with \acrshort{ros}2 using \textbf{rosbridge} (through \textbf{websockets}), a user can define where the robots initially are and where they should go, and by pressing only a button they will start moving.   

\subsection{Drones control}

This project aims to control the drones \textbf{autonomously}: in order to do that the drones are equipped with an \textbf{autopilot module} called \textit{Pixhawk 4}. In such a manner you do not need to control the drones \textbf{manually} (i.e., setting the motors speed), but you can just send commands (e.g. move forward or backward, turn left or right) and the autopilot will do the rest. Luckily, there is a \textbf{bridge} that permits the communication between this module and \acrshort{ros}2, so you are able to create subscribers and publishers nodes that interface \textbf{directly} with \textbf{PX4 UORB topics}\cite{px4}.

The main command that is continuously sent is called \textbf{offboard}: it let the drone know there is still someone that wants to control it, otherwise it will land, for security reasons. The other one is \textbf{TrajectorySetPoint} which lets you set the goal position and orientation of the drone; then, initial and final position are \textbf{interpolated} with a \textbf{Catmull-Rom spline}, in order to have a smoother trajectory.

The drone uses \textbf{GPS} to know where it is, but for testing purposes in the real world, the room chosen could not provide the necessary signal, so it was used \textbf{OptiTrack cameras} to simulate it. Thanks to some \textbf{reflective surfaces}, the eight cameras can \textbf{estimate} its position and orientation, providing a \textbf{temporary alternative} to GPS data, but it works only inside a limited space.

But, before testing the drone in the real world, with possible catastrophic consequences, it is important to be sure that it works in a \textbf{simulation} at least; here Gazebo\footnote{A simulation suite, will be described as well in \autoref{sec:gazebo}} comes to help. When testing with Gazebo, only the position information could be read directly from the interface provided by the program itself, while the orientation one comes from the controller bridge: these data are then fused together inside a node and sent to the drone to give it feedback of its actions\footnote{This is called \textbf{odometry}}. Speaking about the real environment, if GPS is available it will be possible to obtain position and orientation information directly from the drone, so the setup is quite the same. 

With everything put together, the drone can be controlled autonomously.

\subsection{Thesis outline}

This thesis is structured as follows:
\begin{itemize}
  \item \textbf{\autoref{cha:techstack}} describes software and libraries employed in the development of the system, for a better understanding;
  \item \textbf{\autoref{cha:realworld}} shows the steps needed to implement the system with real robots;
  \item \textbf{\autoref{cha:simworld}} covers the simulation part, developing both a simulated environment and a robot;
  \item \textbf{\autoref{cha:navigation}} introduces the navigation system, and shows how it is applied to this project;
  \item \textbf{\autoref{cha:planningbridge}} describes how the planning part was integrated with everything else, with the help of custom bridge nodes;
  \item \textbf{\autoref{cha:futureworks}} discusses possible further developments and improvements.
\end{itemize}