\addcontentsline{toc}{chapter}{Abstract} % check inglese (?)

\chapter*{Abstract}
\label{abstract}

Robots nowadays are being employed more and more to help humans in their tasks:  from automating a productive chain to helping people in need, from working in dangerous situation to replacing them in repetitive tasks, saving their time and energy. This thesis discusses the design of a surveillance system making use of ground robots, developed in the past few months during my internship at the Robotic Lab in the University of Trento. 

Using the provided robots, a fully functional system has been developed, and it can be used in the university environments, but not only this. Normally if you want to test new algorithms and technics, you should use the real robots, which means physically being in the Lab, and it is not always possible. Indeed, this project comes also as a playground: it can also run in a simulation, and thanks to the fact the simulated environment has been kept as close as possible to the real one, it does not matter where you are testing things out, they should work without any big difference.

In addition to the navigation system, a planning one developed by another student has been integrated, since the entire project is shared: thanks to a web interface it is possible to specify areas the robots should patrol, the available ones will be assigned to the task and they will move autonomously giving their feedback back at the end.