% formato FRONTE RETRO
\documentclass[epsfig,a4paper,11pt,titlepage,twoside,openany]{book}
\usepackage{epsfig}
\usepackage{plain}
\usepackage{setspace}
\usepackage[paperheight=29.7cm,paperwidth=21cm,outer=1.5cm,inner=2.5cm,top=2cm,bottom=2cm]{geometry} % per definizione layout
\usepackage{titlesec} % per formato custom dei titoli dei capitoli

% custom packages import
\usepackage[acronym]{glossaries}

\makenoidxglossaries
\newacronym{ugvs}{UGVs}{Unmanned Ground Vehicles}
\newacronym{ros}{ROS}{Robot Operating System}
\newacronym{uavs}{UAVs}{Unmanned Aerial Vehicles}
\newacronym{nav2}{Nav2}{ROS2 Navigation Stack}
\newacronym{urdf}{URDF}{Unified Robotic Description Format}
\newacronym{sdf}{SDF}{Simulation Description Format}
\newacronym{rviz}{RViz2}{ROS Visualization Tool 2}
\newacronym{amcl}{AMCL}{Adaptive Monte-Carlo Localization}
\newacronym{slam}{SLAM}{Simultaneous Localization and Mapping}
\newacronym{btnav}{BT Navigator}{Behavior Tree Navigator}
\newacronym{ekf}{EKF}{Extended Kalman Filter}
\newacronym{gui}{GUI}{Graphical User Interface}
\newacronym{disi}{DISI}{Dipartimento di Ingegneria e Scienze dell'Informazione}
\newacronym{os}{OS}{Operating System}

\makeatletter
{\catcode`\/=\active
  \gdef\slashbreak{
    \catcode`\/=\active
    \def/{\char`\/\penalty\z@}}
}
\makeatother

\makeatletter
\newcommand\footnoteref[1]{\protected@xdef\@thefnmark{\ref{#1}}\@footnotemark}
\makeatother

\usepackage[dvipsnames]{xcolor}
\usepackage{listings}

\usepackage{filecontents}
\usepackage{graphicx, caption}
\usepackage{wrapfig}
\usepackage{subcaption}

\usepackage{hyperref}

\usepackage{chemarrow}

\usepackage[export]{adjustbox}

\usepackage{tikz}
\tikzset{
  block/.style={rectangle, draw, text width=0.15\textwidth, text centered, rounded corners, minimum height=3em, node distance=0.1\textwidth and 0.25\textwidth},
  inception/.style={rectangle, draw, text width=0.2\textwidth, text centered, rounded corners, minimum height=3em, node distance=0.1\textwidth and 0.25\textwidth},
  arrow/.style={-{Stealth[]}},
  poseserver/.style={rectangle, draw, text width=\textwidth, text centered, rounded corners, minimum height=3em, node distance=0.1\textwidth},
  navigationclient/.style={rectangle, draw, text width=, text centered, rounded corners, minimum height=3em, node distance=0.1\textwidth},
  }
\usetikzlibrary{positioning,arrows.meta,fit}
\usepackage{multicol}

\newcommand{\arguments}[2]{\textbf{#1}: \textit{#2}}

\newif\ifapplymulticol
\makeatletter
\lst@AddToHook{InitVars}{\ifapplymulticol\edef\lst@next{\noexpand\multicols{2}}\expandafter\lst@next\fi}
\lst@AddToHook{ExitVars}{\ifapplymulticol\def\lst@next{\global\let\@checkend\@gobble
                      \endmulticols
                      \global\let\@checkend\lst@@checkend}
        \expandafter\lst@next\fi}
\makeatother

\def\code#1{\texttt{#1}}

\setlength\intextsep{0pt}

\lstdefinelanguage{json}{
    morekeywords=[1]{label, center, max, min},
    morekeywords=[2]{povo_1_258},
    literate=
     *{:}{{{\color{Red}{:}}}}{1}
      {,}{{{\color{Red}{,}}}}{1}
      {\{}{{{\color{Blue}{\{}}}}{1}
      {\}}{{{\color{Blue}{\}}}}}{1}
      {[}{{{\color{Blue}{[}}}}{1}
      {]}{{{\color{Blue}{]}}}}{1}
}

\lstdefinestyle{jsonStile}{
    frame=tb,
    language=json,
    columns=flexible,
    keepspaces=true,
    breaklines=true,
    basicstyle=\small\ttfamily,
    keywordstyle=[1]\color{Green},
    keywordstyle=[2]\color{NavyBlue},
}

\definecolor{ipython_red}{RGB}{186, 33, 33}
\definecolor{ipython_green}{RGB}{0, 128, 0}
\definecolor{ipython_cyan}{RGB}{64, 128, 128}
\definecolor{ipython_purple}{RGB}{170, 34, 255}

\lstdefinestyle{PythonStile}{
    frame=tb,
    language=Python,
    columns=flexible,
    keepspaces=true,
    breaklines=true,
    identifierstyle=\color{black},
    commentstyle=\color{ipython_cyan},
    stringstyle=\color{ipython_red},
    basicstyle=\small\ttfamily,
    keywordstyle=[1]\color{ipython_green},
    keywordstyle=[2]\color{ipython_purple}
}

\lstdefinelanguage{Python}{
    morekeywords=[1]{open, load, dump, dirname, abspath, \_\_file\_\_, import, array, cos, sin, dot, dict},
    sensitive=true,
    morekeywords=[2]{translation, rot\_z, scale\_y, current\_folder, original\_name, dict\_name, original\_data, rot\_matrix, scale\_matrix, roto\_scaling, dict\_data, obj, k, v},
    morecomment=[l]\#,
    morestring=[b]",
}

\lstdefinelanguage{Dockerfile}
{
  morekeywords={FROM, RUN, CMD, LABEL, MAINTAINER, EXPOSE, ENV, ADD, COPY,
    ENTRYPOINT, VOLUME, USER, WORKDIR, ARG, ONBUILD, STOPSIGNAL, HEALTHCHECK,
    SHELL},
  morecomment=[l]{\#},
  morestring=[b]',
  morestring=[b]",
}

\lstdefinestyle{DockerStyle}{
    frame=tb,
    language=Dockerfile,
    columns=flexible,
    keepspaces=true,
    showstringspaces=false,
    basicstyle=\small\ttfamily,
    commentstyle=\color{Gray},
    keywordstyle=\color{Red},
    stringstyle=\color{RoyalBlue},
}

\lstdefinestyle{CStyleTextWidth}{
    language=C,
    columns=flexible,
    keepspaces=true,
    breaklines=true,
    linewidth=\textwidth,
    showstringspaces=false,
    morekeywords=[2]{canTransform, lookupTransform},
    basicstyle=\small\ttfamily,
    commentstyle=\color{Gray},
    keywordstyle=\color{Red},
    keywordstyle=[2]\color{Purple},
    stringstyle=\color{RoyalBlue},
}

\lstdefinestyle{CStyleNoFrame}{
    language=C,
    columns=flexible,
    keepspaces=true,
    breaklines=true,
    showstringspaces=false,
    morekeywords=[2]{send_goal},
    basicstyle=\small\ttfamily,
    commentstyle=\color{Gray},
    keywordstyle=\color{Red},
    keywordstyle=[2]\color{Purple},
    stringstyle=\color{RoyalBlue},
}

\lstdefinestyle{CStyle}{
    language=C,
    frame=tb,
    columns=flexible,
    keepspaces=true,
    breaklines=true,
    showstringspaces=false,
    morekeywords=[2]{this, GOAL_INDEX, room_goal_, goal_position_, future, result, get_pose, continue_work, progress_},
    basicstyle=\small\ttfamily,
    commentstyle=\color{Gray},
    keywordstyle=\color{Red},
    keywordstyle=[2]\color{Purple},
    stringstyle=\color{RoyalBlue},
}

\lstdefinelanguage{xml}{
  morestring=[b]",
  morecomment=[s]{<?}{?>},
  morekeywords={type,pkg,name,args}
}

\lstdefinestyle{xmlStyle}{
    frame=tb,
    language=XML,
    basicstyle=\ttfamily,
    columns=flexible,
    showstringspaces=false,
    commentstyle=\color{gray}\upshape,
    stringstyle=\color{ForestGreen},
    identifierstyle=\color{Red},
    keywordstyle=\color{Black}
}

\lstdefinelanguage{bash}{
    morestring=[b]",
    morecomment=[l]{\#}
}

\lstdefinestyle{bashStyle}{
    frame=tb,
    language=bash,
    basicstyle=\ttfamily,
    columns=flexible,
    showstringspaces=false,
    commentstyle=\color{gray}\upshape,
    stringstyle=\color{Blue},
    keywordstyle=\color{Black}
}

%%%%%%%%%%%%%%
% supporto lettere accentate
%
%\usepackage[latin1]{inputenc} % per Windows;
\usepackage[utf8x]{inputenc} % per Linux (richiede il pacchetto unicode);
%\usepackage[applemac]{inputenc} % per Mac.

\singlespacing

\usepackage[english]{babel}

\begin{document}
% \begin{slashbreak}

  % nessuna numerazione
  \pagenumbering{gobble} 
  \pagestyle{plain}

\thispagestyle{empty}

\begin{center}
  \begin{figure}[h!]
    \centerline{\psfig{file=marchio_unitrento_colore_it_202002.eps,width=0.6\textwidth}}
  \end{figure}

  \vspace{2 cm} 

  \LARGE{Dipartimento di Ingegneria e Scienza dell’Informazione\\}

  \vspace{1 cm} 
  \Large{Corso di Laurea in\\
    Informatica
  }

  \vspace{2 cm} 
  \Large\textsc{Elaborato finale\\} 
  \vspace{1 cm} 
  \Huge\textsc{Drone Surveillance\\}
  \Large{\it{A ROS2 based solution for UGV navigation}}


  \vspace{2 cm} 
  \begin{tabular*}{\textwidth}{ c @{\extracolsep{\fill}} c }
  \Large{Supervisore} & \Large{Laureando}\\
  \Large{Palopoli Luigi}& \Large{Franzin Mattia}\\
  \end{tabular*}

  \vspace{2 cm} 

  \Large{Anno accademico 2021/2022}
  
\end{center}



  \cleardoublepage
 
  \input{ringraziamenti}
  \clearpage
  \pagestyle{plain} % nessuna intestazione e pie pagina con numero al centro
  
  % inizio numerazione pagine in numeri arabi
  \mainmatter
    % indice
    \tableofcontents
    \clearpage
    
    
          
    % gruppo per definizone di successione capitoli senza interruzione di pagina
    \begingroup
      % nessuna interruzione di pagina tra capitoli
      % ridefinizione dei comandi di clear page
      % \renewcommand{\cleardoublepage}{} 
      % \renewcommand{\clearpage}{} 
      % redefinizione del formato del titolo del capitolo
      % da formato
      %   Capitolo X
      %   Titolo capitolo
      % a formato
      %   X   Titolo capitolo
      
      \titleformat{\chapter}
        {\normalfont\Huge\bfseries}{\thechapter}{1em}{}
        
      \titlespacing*{\chapter}{0pt}{0.59in}{0.02in}
      \titlespacing*{\section}{0pt}{0.20in}{0.02in}
      \titlespacing*{\subsection}{0pt}{0.10in}{0.02in} 
      
      \addcontentsline{toc}{chapter}{Abstract} % check inglese (?)

\chapter*{Abstract}
\label{abstract}

Robots nowadays are being employed more and more to help humans in their tasks:  from automating a productive chain to helping people in need, from working in dangerous situation to replacing them in repetitive tasks, saving their time and energy. This thesis discusses the design of a surveillance system making use of ground robots, developed in the past few months during my internship at the Robotic Lab in the University of Trento. 

Using the provided robots, a fully functional system has been developed, and it can be used in the university environments, but not only this. Normally if you want to test new algorithms and technics, you should use the real robots, which means physically being in the Lab, and it is not always possible. Indeed, this project comes also as a playground: it can also run in a simulation, and thanks to the fact the simulated environment has been kept as close as possible to the real one, it does not matter where you are testing things out, they should work without any big difference.

In addition to the navigation system, a planning one developed by another student has been integrated, since the entire project is shared: thanks to a web interface it is possible to specify areas the robots should patrol, the available ones will be assigned to the task and they will move autonomously giving their feedback back at the end.
      \chapter{Introduction} % senza numerazione
\label{cha:intro}

\section{Summary}

The goal of this project, as described in the abstract, aims to develop a complete system for \textbf{surveillance proposes} using ground robots, but also flying ones.
Since the project is quite complex, it is \textbf{shared} among other two students, with each one of us focusing into a specific subtask. It is constituted of: 

\begin{itemize}
  \item a \textbf{planning system}, responsible for dispatching the surveillance tasks to \acrfull{ugvs} and/or \acrfull{uavs} the best possible way, based on their battery charge, giving them some waypoints to reach 
  \item a solution to autonomously control the \textbf{drones} (\acrshort{uavs})
  \item a solution to autonomously control ground robots avoiding \textbf{dynamic obstacles} (\acrshort{ugvs})
\end{itemize}

Everything is developed using \acrshort{ros}2, that gives us the possibility to work \textbf{separately} on your own project, and thanks to \acrshort{ros} \textbf{modularity}, at the end we only need to put everything together.

The reason behind my choice about the internship and this consequent thesis is leaded by my \textbf{growing interest} in this topic: just some month before taking part in this work, I attended a course about \textit{robotic fundamentals}\cite{intro2robotics} where we also developed a project quite similar to this, but with a robotic manipulator.

For me, is the possibility to give to some inanimate object, like a wheeled robot or a manipulator, something that could be described as \textbf{intelligence}, the ability to perform some tasks in response of other ones, figuring out which ones are the \textbf{best suitable} for each particular situation. Some challenges I decided to took are: the choice of the \textbf{new version} of ROS, with some great improves respect to the older one, but with less community support, and also working for the first time on a \textbf{mobile robot}.

\section{Other projects involved}

What follows is a brief description of the work done by the other two students to have a better and clear idea of the workflow of the entire system.

\subsection{Planning system, fleet management and web interface}
\label{sub:planning}

It is brain of the system, it lets the user choose which type of robot he wants to use (only ground robots or drones with ground robots as auxiliary), and then it dispatches the tasks to the right robots. It is possible to define multiple goals for each robot, and once done, the planner will start figuring out the best way to organize the fleet.

In order to resolve a problem it must have been defined a \textbf{domain}: this contains a description of space you are interested in, in which you could specify \textbf{reachable targets}, making use of a set of \textbf{predefined actions}, that requires some \textbf{preconditions} to be met before executing and causes some \textbf{effects} to the environment. In order to solve a problem, we need to check if its domain is \textbf{compatible} with the planning one, or in other words, if the goal is achievable, and if so, it is possible to look for a \textbf{solution} (i.e., a sequence of actions to be performed).

Each robot waits for new commands to be received, with one specific implementation for each type of task. Here follows a list of actions currently available:

\bigskip

\begin{minipage}[h]{0.45\textwidth}
  \centering
  \textbf{\acrshort{uavs} actions}
  \begin{itemize}
    \centering
    \item \code{uav\_move}
    \item \code{uav\_take\_photo}    
    \item \code{uav\_land\_on\_ugv}
    \item \code{uav\_take\_off\_ugv}
  \end{itemize}
\end{minipage}
\begin{minipage}[h]{0.45\textwidth}
  \centering
  \textbf{\acrshort{ugvs} actions}
    \begin{itemize}
      \centering
      \item \code{ugv\_move}
      \item \code{ugv\_charge}
      \item \code{try\_ugv\_charge}
      \item \code{ugv\_transporting\_uav\_move}
    \end{itemize}
\end{minipage}

\bigskip

These actions are the only ones the planner uses when a problem is specified to find a possible outcome: the output are \textbf{waterfall actions} that should be done, so when one is completed it is possible to move to the next one. The implementation is left to whom works on that robot, because of their \textbf{better understanding} of that particular robot.

But, speaking about movement, we need some \textbf{coordinates}. These are extracted from the 3D mesh and thanks to an \textbf{annotation program} it is possible to assign them a name: in this way instead of using a bunch of numbers, we use the name of the room, and their relationship is defined in a file. Each room is then \textbf{connected} to the others by default.

Thanks to a web interface, communicating with \acrshort{ros}2 using \textbf{rosbridge} (it makes use of \textbf{websockets}) (?), a user can define where the robots initially are and where they should go, and by pressing only a button they will start moving.   

\subsection{Drones control}

This project aims to control the drones \textbf{autonomously}: in order to do that the drones are equipped with an \textbf{autopilot module} called \textit{Pixhawk 4}. In such a manner you do not need to control the drones \textbf{manually} (i.e., setting the motors speed), but you can just send commands (e.g. move forward or backward, turn left or right) and the autopilot will do the rest. Luckily, there is a \textbf{bridge} that permits the communication between this module and \acrshort{ros}2, so you are able to create subscribers and publishers nodes that interface \textbf{directly} with \textbf{PX4 UORB topics}\cite{px4}.

The main command that is continuously sent is called \textbf{offboard}: it let the drone know there is still someone that wants to control it, otherwise it will land, for security reasons. The other one is \textbf{TrajectorySetPoint} which lets you set the goal position and orientation of the drone; then, initial and final position are \textbf{interpolated} with a \textbf{Catmull-Rom spline}, in order to have a smoother trajectory.

The drone uses \textbf{GPS} to know where it is, but for testing purposes in the real world, the room chosen could not provide the necessary signal, so it was used \textbf{OptiTrack cameras} to simulate it. Thanks to some \textbf{reflective surfaces}, the eight cameras can \textbf{estimate} its position and orientation, providing a \textbf{temporary alternative} to GPS data, but it works only inside a limited space.

But, before testing the drone in the real world, with possible catastrophic consequences, it is important to be sure that it works in a \textbf{simulation} at least; here Gazebo\footnote{A simulation suite, will be described as well in \autoref{sec:gazebo}} comes to help. When testing with Gazebo, only the position information could be read directly from the interface provided by the program itself, while the orientation one comes from the controller bridge: these data are then fused together inside a node and sent to the drone to give it feedback of its actions\footnote{This is called \textbf{odometry}}. Speaking about the real environment, if GPS is available it will be possible to obtain position and orientation information directly from the drone, so the setup is quite the same. 

With everything put together, the drone can be controlled autonomously.

\subsection{Thesis outline}

This thesis is structured as follows:
\begin{itemize}
  \item \textbf{\autoref{cha:techstack}} describes software and libraries employed in the development of the system, for a better understanding;
  \item \textbf{\autoref{cha:realworld}} shows the steps needed to implement the system with real robots;
  \item \textbf{\autoref{cha:simworld}} covers the simulation part, developing both a simulated environment and a robot;
  \item \textbf{\autoref{cha:navigation}} introduces the navigation system, and shows how it is applied to this project;
  \item \textbf{\autoref{cha:planningbridge}} describes how the planning part was integrated with everything else, with the help of custom bridge nodes;
  \item \textbf{\autoref{cha:futureworks}} discusses possible further developments and improvements.
\end{itemize}
      \chapter{About the technology stack}
\label{cha:techstack}

Before describing the entire project, we need to know what we are talking about. In particular, there are the softwares or libraries strongly used in the project:  
\begin{itemize}
    \item Docker
    \item ROS2
    \item Gazebo
    \item Navigation stack, nav2
    \end{itemize}

\section{Docker}

% insert introduction
Docker was used (impiegato?) because of two reasons:
\begin{itemize}
    \item permit (?) development on any computer not running necessarily Ubuntu 20.04 (?) as its OS
    \item keep the workspace equal for every device
    \item separate current workspace from all the other ones, to avoid conflicts or dependencies problems
\end{itemize}  

Since there are multiple people working on this robot with their own projects, it is a bad practice to mix everything together. % continue

Indeed, the entire project comes with a Dockerfile with the needed setup and packages required to develop and run the real robot or the simulation; if someone wants, (?) could still install everything by hand\footnote{See attachment for the dependencies required or Dockerfile attachment} on its own system.


\section{\acrshort{ros}2}

% https://www.allaboutcircuits.com/technical-articles/an-introduction-to-robot-operating-system-ros/#:~:text=ROS%20is%20designed%20to%20be,using%20the%20publish%2Fsubscribe%20model., 04/08
% https://docs.ros.org/en/foxy/index.html, 27/07

{\it \acrshort{ros}2, which stands for Robot Operative System 2, is a set of software libraries and tools for building robot applications.}\cite{ros2desc} On it, every process is a node and each node has the possibility to talk to other ones thanks to some channel of communications called topics. How? Using the publish/subscribe model.

ROS is an OS in concept because it provides all the services that any other OS does—like hardware abstraction, low-level device control, implementation of commonly-used functionality, message-passing between processes, and package management.\footnote{allaboutcircuits.com, 04/08}

In other words, executing a ROS node on your computer is the same of executing it on a robot, as long as you are using the same external devices. Or maybe, you don't need to.

\section{Gazebo}

Here comes the simulation part, because thanks to Gazebo-ROS integration it is possible, with some additional libraries, to simulate external devices. It is not limited to this, because it gives you a sandbox you could set up as the real environment you want to simulate.

In this way, you don't need to have the real robot next to you, you can just use a virtual one: it's very useful in the case you want to change some algorithms or parameters, and you need to test them even if you are on the opposite side of the world.

% includere il fatto che l'ho fatto per questo motivo la parte di simulazione?

\section{rviz2}

Rviz2, is a visualization tool for ROS: in this way, you are able to visualize the sensors, the robots, the environment, and also the algorithms; in general, you can see the actual state of your \acrshort{ros} system.

\section{\acrfull{nav2}}

The Navigation Stack is a collection of packages of \acrshort{ros}2 which provides useful tools helping you with the navigation of your robot: path planning, localization, and so on. You can use it by passing your desired configurations in a YAML file which will be used when you launch the navigation stack: basically, you specify what nodes you want to use and the parameters you want to set, and everything is done for you. (?)
These packages were used as a base to everything else, so in this way I don't need to worry about the navigation algorithms, since my goal is to develop a system to move around ground robots where indicated.

\begin{figure}[h]
    % \noindent\makebox[0.9\textwidth]{\includegraphics[width=0.9\paperwidth]{images/nav2_architecture}}
    \centering
    \includegraphics[width=0.8\textwidth]{images/nav2_architecture}
    \caption{Navigation architecture}
  
  \end{figure}

      % Parlare del tipo di robot, diff drive, ecc. (?)

\chapter{Working in a real environment}
\label{cha:realworld}

The first thing we need is the robot: this projects uses the one shown in \autoref{fig:shelfino}, and it comes from \acrfull{disi} of University of Trento.

\bigskip

\begin{figure}[h]
  \centering
  \includegraphics[width=0.5\textwidth]{images/shelfino}
  \caption{The, so called, \textit{shelfino (one)} robot}
  \label{fig:shelfino}
\end{figure}

\section{Shelfino setup} 

\subsection{Adapting existing code}

The \textbf{hardware interface} had already been developed by some researchers in recent years, and they are now no longer here. The original idea was to develop some ROS nodes that would connect to \textbf{encoders}, \textbf{motors drivers} and \textbf{lidars}, but I was not given specific information about brands or manufacturers of some of these components.

Another attempt was to rewrite the code in ROS2: it was not straightforward, and in the end it did not work, since the libraries used refer to a network infrastructure that I am not well aware of. In addition, some features have not been ported to \acrshort{ros}2 \textbf{in the same way}, leading to some potential \textbf{differences} from the original code.

\subsection{\acrshort{ros}1 Bridge}

The final attempt involved the use of a \acrshort{ros}2 package called \textit{ros1\_bridge}. Because both \acrshort{ros}1 and \acrshort{ros}2 use their \textbf{local network} to deliver messages, you can create a node that \textbf{listens to both networks} and when something is received from one side, it is sent to the other: this way you can run the \textbf{existing code as it is} on a \acrshort{ros}1 node (with its dependencies), while making use of its topics and services on a \acrshort{ros}2 workspace at the same time.

But, in order to run both \acrshort{ros}1 and \acrshort{ros}2 nodes on the same machine (or Docker container), they need the same version of Ubuntu. \textbf{\acrshort{ros}2 Foxy} version is mandatory due to planning libraries, and runs only on \textbf{Ubuntu 20.04}, so also \acrshort{ros}1 must use it; the original code, however, was written for the previous version (\acrshort{ros}1 Melodic, running only on Ubuntu 18.04), but furtunately it works smoothly on the new one\footnote{The biggest change was introducing the support for Python 3}, \textbf{\acrshort{ros}1 Noetic} (for the same version of the \acrshort{os}).

To be more precise, the \code{dynamic\_bridge} of this package was used, instead of \code{parametric\_bridge} and \code{static\_bridge}: with the first one you can choose what topics or services you want to bridge, but it has some bugs and does not work well, while the second one must be compiled whenever there are some changes; the dynamic one, instead, adapts to every occasion.

\subsection{Existing nodes explained}
\label{subsec:nodes}

In order to start the existing nodes, a custom launch file was created. A launch file is a file containing information about \textbf{which nodes}, possibly with some \textbf{parameters}, must be \textbf{started} when the system is launched, instead of doing it one by one. Three nodes are inside this launch file:

\begin{itemize}
    \item \code{lidar\_position}
    \item \code{hw\_interface}
    \item \code{odom\_node}
\end{itemize}

\subsubsection{lidar\_position} % forse togliere e forse  mettere 0.43 (?), dire che è stata fatta su ros2 ora

\begin{lstlisting}[
  label={lst:lidarpos},
  language=xml,
  style=xmlStyle
  ]
  <node pkg="tf" type="static_transform_publisher" name="lidar_position"
    args="0 0 0.45 0 0 0 base_link base_laser 10" />
\end{lstlisting}

This is a node (with a custom name) of the \acrshort{ros}1 \code{tf} package, \textit{a package that lets the user keep track of multiple coordinate frames over time [...] and lets the user transform points, vectors, etc. between any two coordinate frames at any desired point in time}.\cite{tf} The node executable is called \code{static\_transform\_publisher} and it is used to \textit{publish a static coordinate transform to tf using an x/y/z offset in meters and roll/pitch/yaw in radians [...]. The period, in milliseconds, specifies how often to send a transform}. \cite{tf} 
As \autoref{lst:lidarpos} shows, here it is used to describe the \textbf{transformation} between \code{base\_link}, that is the root frame (located on the ground under the robot, in the origin), and \code{lidar\_link} (0.45 meters above) every \code{10ms}. So when some points are returned by the laser scans, we also know the height, not otherwise specified, since it is a 2D scan.

\subsubsection{hw\_interface}

It is a node designed to interface with the \textbf{underlying hardware}, thanks to the custom \code{hardwareglo\-balinterface} library previously developed by researchers. It is a \textbf{helper class} that uses a \textbf{ZeroMQ framework} to send messages between the hardware (client) and the server running on the BeagleBone, which keeps track of the data received and makes them accessible using class methods.

Basically, using the interface just described, it reads data from \textbf{encoders}, \textbf{lidar} and \textbf{tracking camera} and makes them available as \acrshort{ros}2 topics (\code{/encoders, /scan, /t265}). It also subscribes to a \code{/cmd\_vel} topic from which it reads the desired \textbf{linear} and \textbf{angular velocities} and dispatches them to the robot's motors accordingly.

\subsubsection{odom\_node}

This node uses information received from \textbf{encoders topic} about wheel rotation and \textbf{tracking camera topic}\footnote{A tracking camera is used in addition to wheel rotation to prevent odometry drift. If you do not use a tracking camera, when the robot is facing a wall and the wheels continue to rotate (obviously the robot will not move), odometry will tell the opposite, that it is still moving. When using it, if the robot moves, certainly what the tracking camera perceives changes, otherwise it will not, and it can be used to avoid the above.} to calculate the \textbf{odometry}, which represents an estimate of the \textbf{position} and \textbf{rotation} of the robot from which it started. This new information is then published both as a message in a topic, and as a transformation between \code{odom\_frame} and \code{base\_link}, employed by the \textbf{navigation system} as described in \autoref{cha:navigation}.
      \chapter{Working in a simulated environment}

The choice to develop a \textbf{simulation} is due to the fact that sometimes you may not physically be in the university to use the real robot. The challenge here is doing a good job with the \textbf{abstraction}, which means that the simulated environment and the simulated robot should be \textbf{as close as possible} to real ones. In other words, everything that was working before (i.e. navigation and planning) should now continue to work even if you are no longer in the real world; in theory you should not notice if you are in the real world or in a simulated one.

\section{Models}

In order to create a complete simulation, we need to create what is missing from the reality, which are the models of the environment and the robot.

\subsection{Environment}
\label{sub:map}

A mesh of the Povo upper floors was provided by professor \textit{Marco Roveri} \cite{roveri}. Unfortunately, when compared to the navigation map generated from real data, it turns out that the mesh is \textbf{not very accurate}. Two workarounds are possible: creating a \textbf{new navigation map} for simulation purposes or \textbf{adapting the mesh} as much as possible (?); it turns out, as described in \autoref{sub:waypoints}, by scaling the mesh \textbf{1.0925 times} in the long direction leads to a better result. % testare slam con mesh e fare nuova mappa
Then a model is creating using \acrfull{sdf}, with the mesh employed in \textbf{visual} and \textbf{collisions} tags (?); this model is then used in \code{world} file\footnote{Always written in \acrshort{sdf}} (alongside with ground plane and sun ones, provided by Gazebo and the robot model, of course).

\begin{figure}[h]
    \centering
    \includegraphics[width=0.8\textwidth]{images/3d\_povo\_model}
    \caption{Povo model for simulation. Colors were used for a better visualization.}
\end{figure}

\subsection{Robot}

Part of the setup is inspired to Automatic Addison tutorials\cite{tutorials}. % forse togliere (?)

The robot mesh was created starting from \textit{shelfino} (see \autoref{fig:shelfino}), trying to keep it \textbf{as similar as possible}, and each piece (robot base, lidar, wheels and caster wheel) was exported singularly. For performance reasons in Gazebo, the only mesh actually used is the robot base, because of its special structure to \textbf{avoid contact} with the spinning wheels \textbf{partially inside}. The other ones are substituted by \acrshort{sdf} or \acrfull{urdf} built-in geometric shapes (i.e. cylinders).
Both \acrshort{sdf} and \acrshort{urdf} were used: the first one is used by Gazebo, the second one by a node called \code{robot\_state\_publisher} that helps visualize the robot in \acrshort{rviz}.

\begin{figure}[h]
    \centering
    \includegraphics[width=0.7\textwidth]{images/shelfino_3d.png}
    \caption{Shelfino model for simulation.}
\end{figure}

\section{Gazebo plugins}

After placing the models in Gazebo, we need some way to make our robot \textbf{interact} with the environment. It is no longer possible to utilize the nodes in \autoref{subsec:nodes} because they make use of a real hardware, but thankfully Gazebo provides some \textbf{plugins} that can be added to the \acrshort{sdf} model as a normal tag.
We always need a way to set \textbf{linear and angular velocities} to make the robot moves as desired, a way to \textbf{calculate odometry} and a lidar to \textbf{detect obstacles}. Once done, the robot can be used as if it were a real robot.

Sadly, no tracking camera plugins are available, so we content ourselves with only wheel rotation information to calculate odometry. 

\subsection{Differential drive plugin}

Once added, some configurations needs to be made:
\begin{itemize}
    \item set \textbf{update rate} (Hz)
    \item specify \textbf{name} of wheel joints (control)
    \item set wheels \textbf{separation} and \textbf{diameter} (kinematics)
    \item set max \textbf{torque} and \textbf{acceleration} (limits)
    \item specify topic names from where \textbf{receiving desired velocity} and \textbf{publish odometry data}
    \item whether to \textbf{publish frame transformations} and their names (used by \acrshort{rviz})
\end{itemize}
After that, the robot can now move.

\subsection{Joint state publisher plugin} % forse toglierlo (?)

This plugin is used to know how much the wheels are spinning in order to visualize them in \acrshort{rviz}. Only update rate and wheel names are required. (?)

\subsection{Ray sensor plugin}

To be able to detect obstacles, a lidar is needed. To do so, you can just use the \textbf{sensor} tag \code{ray} of Gazebo, setting \textbf{scan}, \textbf{range} and \textbf{noise} information, but also a plugin is required to make it work. Only configurations needed are:
\begin{itemize}
    \item what \acrshort{ros} message type to use as \textbf{output} (\code{sensor\_msgs/LaserScan})
    \item frame name where the lidar is \textbf{attacched} to (described somewhere else in the model)
\end{itemize}

% add robot description of link, joints, ecc...? o qualcos'altro? tipo sdf come allegato? (?)

% aggiungere struttura cartella con models e worlds solo
      % Parlare del tipo di robot, diff drive, ecc. (?), sistemare parte ground_robot_navigation, spiegando in generale, non nello specifico

\chapter{Navigation}
\label{cha:navigation}

For the navigation part it has been used the \acrshort{nav2} package, as described in \autoref{cha:techstack}. As said, the only thing you needed is passing a \textbf{YAML config file} to \code{navigation\_bringup.py} script and it will start up the nodes with the desired parameters set.

\section{\acrshort{nav2} nodes introduction}

This is a brief description of the nodes responsible for the navigation, in order to have a better idea of the \textbf{workflow}. It is a summary of \code{README} files coming from their \textbf{GitHub} repository \cite{nav2github}. In \autoref{fig:nav2} you can see the \acrshort{nav2} architecture.

\bigskip

\begin{figure}[h]
    \centering
    \includegraphics[width=0.8\textwidth]{images/nav2_architecture}
    \caption{Navigation package overview}
    \label{fig:nav2}
\end{figure}

\subsection{Map Server}

This package is used to provide \textbf{map} functionalities to \acrshort{ros}. Basically, it gives you the possibility to \textbf{load} a map from a file and use it for \acrshort{amcl}, or \textbf{save} it after \acrshort{slam} has been run.

\subsection{\acrfull{amcl}}

\acrshort{amcl} is used to \textbf{localize} the robot on a known \textbf{map} using a 2D laser scanner \textbf{probabilistically}. Firstly, once the map is loaded, the initial robot pose \textbf{must be set}\footnote{\acrshort{rviz} lets you set the initial robot pose graphically}: what is going on under the hood is publishing a \code{map} $\rightarrow$ \code{odom} transformation so that the \code{odom} $\rightarrow$ \code{base\_link} one leads to the \textbf{real position} of the robot.

\subsection{\acrfull{slam}}

It is a collection of techniques used to \textbf{localize} and \textbf{map} the environment \textbf{simultaneously}, using 2D laser scans \cite{slam}. Once the map is completely built, it is possible to save it to a file and pass it to the Map Server.

\subsection{Recoveries Server}

It is responsible for executing simple controlled robot movements, like backing up, rotating and stopping when \textbf{recovery} is needed, like when the robot \textbf{hits} something.

\subsection{\acrfull{btnav}}

This one makes use of \textbf{behavior trees} to define the actions to be performed when the robot is in a certain \textbf{state}: for example, when no problem has met, it will continue to navigate and reach the current goal, but when it hits something, the behavior tree will execute the recovery action.

\begin{figure}[h]
    \centering
    \includegraphics[width=0.9\textwidth]{images/bt-alpha.png}
    \caption{Currently used behavior tree. Groot lets you visualize it, also in real-time. \cite{groot}}
\end{figure}

\subsection{Planner Server}

It implements behavior trees to \textbf{compute path to pose}; it is also possible to choose which pathfinding algorithm you want to use.

\subsection{Controller Server}

It generates \textbf{command velocities} for the wheels using computed path from Planner Server and send them to the robot.

% \subsection{Waypoint Follower} 

% Instead of just setting a single goal, it is possible to navigate a list of waypoints the robot has to pass through. When a waypoint is reached, some plugins can be attached in order to make the robot take a photo, wait for external command or simply wait. Currently, this feature has not been used in the project, because even if multiple rooms are requested to be visited, the planner will only generate subsequent tasks, passing them one by one, after the previous one has been completed.

\subsection{Lifecycle Manager}

All the previous nodes interface with \textbf{Lifecycle Manager}, which brings \acrshort{ros}2 \textbf{managed nodes}\footnote{\textit{A managed life cycle for nodes allows greater control over the state of ROS system. [...] a managed node presents a known interface, executes according to a known life cycle state machine, and otherwise can be considered a black box.} \cite{lifecycle}} concept to the navigation stack: in such a manner, before nodes begin their execution, it \textbf{checks} if all of them were \textbf{launched correctly} and are ready to start, to avoid unexpected behaviors.

\begin{figure}[h]
    \centering
    \includegraphics[width=0.8\textwidth]{images/uml_lifecycle_manager}
    \caption{Sequence of service calls when startup is requested}
\end{figure}

\section{\code{ground\_robot\_navigation} package integration}

\begin{wrapfigure}{l}{0.35\textwidth}
    \includegraphics[width=0.35\textwidth]{images/nav_folder}
    \caption{Navigation folder}
\end{wrapfigure}

Everything required to navigating through the environment is provided by \code{ground\_robot\_navigation} package. The complete structure is shown in \autoref{fig:folder}

% TODO: sistemare folder structure (togliere v2 da urdf e aggiugnere "..." se una cartella contiene altri file; sistemare altezza lidar? 0.45 invece di 0.41 come su shelfino?)

Regardless if the robot is real or simulated, there are always a \code{scan}, \code{odom} and \code{cmd\_vel} topic and a \code{map}, \code{odom} and \code{base\_link}\footnote{All the other links of the robot are connected to the last one, the root one} frame. What is still missing is a way to pass this information to \acrshort{nav2} and here comes \code{ground-robot} package, designed specifically for this project.

Inside \code{\acrshort{rviz}} folder there are two \textbf{configuration files} which are going to be loaded when the program starts executing: in the \textit{\_real} one, only some link/joint names are \textbf{missing} because they are no longer simulated and there are no such frame transformations. % check se ne basta uno o no (?)

Folders like \code{include} or \code{src} are left even if empty in the project folder, because it is possible to \textbf{integrate custom algorithms} inside the \acrlong{nav2}, thanks to \code{nav2\_core} package. The same applies for \code{config} one, where you can put your own configuration files, how it was done originally when using \acrfull{ekf} to estimate robot odemetry\footnote{Later removed because of its buggy behavior}, fusing multiple sensors.

\code{params} folder contains the parameters file described in the beginning of \autoref{cha:navigation}.

Inside the \code{launch} folder there is the script used to \textbf{start} the entire \textbf{navigation system}; normally it launches the version with \textbf{real} environment and robot. Here follows a list of the possible arguments you can pass as $<$name$>$:=$<$value$>$ pairs: % (?)

\begin{itemize}
    \item \arguments{use\_namespace/namespace}{whether to apply a namespace and its name}
    \item \arguments{autostart}{if \acrshort{nav2} should be automatically started}
    \item \arguments{slam}{whether to run \acrshort{slam}}
    \item \arguments{params\_file}{full path to parameters file to use for navigation nodes}
    \item \arguments{bt\_xml\_filename}{full path to the behavior tree xml file to use}
    \item \arguments{map}{full path to map file to load}
    \item \arguments{use\_simulator}{whether to start the simulator}
    \item \arguments{headless}{whether to execute Gazebo client \acrshort{gui}}
    \item \arguments{use\_sim\_time}{whether to use simulation time, starting from zero}
    \item \arguments{use\_robot\_state\_pub}{whether to start the robot joint state publisher}
    \item \arguments{world}{full path to the world model file to load}
    \item \arguments{use\_rviz}{whether to start \acrshort{rviz}}
    \item \arguments{rviz\_config\_file}{full path to the RVIZ config file to use}
    \item \arguments{model}{full path to robot urdf file for \acrshort{rviz}}
\end{itemize}

\bigskip

If you want to start a simulation, just set \code{use\_simulator} to \textit{true} and change \code{rviz\_config\_file} path; if you want to \textbf{disable} the \acrfull{gui} of Gazebo, set also \textbf{headless} to \textit{true}.

% parlare di velocità lineare e angolare
% odom su simulazione
% sostituire base_laser con lidar_link e provare a lasciare i joint rimossi (?)
      \chapter{Planning integration}

Last, but not least, (?) there is the planning part integration, done by Filippo Rossi\footnote{There will not be any further explanation, since the entire project was already described in \autoref{sub:planning}}. What was described in the previous chapters leads (?) to a working version of the project, but has not any practical use, because you need to set where you want the robot to go. Entertaining but useless (?). This is the goal of this chapter: add some mission the robot needs to complete, without any human interaction directly on the robot (?).

\section{Modifications from original planning}

When the planning was developed, Filippo could not receive any feedback from us, since we had some other challenges to face, like making the robots to move autonomously. So, in response, the planning has come as a general workspace to which we can add some modification to meet our specific needs.

\subsection*{Correct waypoint coordinates}

Thanks to Filippo's waypoints extractor and \code{cvat} (?) software, all the required information could be found in a JSON file: when the robot is told to move in a certain room, the only operation to be done (?) is searching its name in the file and get the corresponding coordinates, but the mesh used for generating those waypoints is not the same of the real environment.
Thanks to Blender (cite (?)), after loading the planning mesh and the map used by the navigation system, I (?) was able to find some modification I can bring to get a better result: turning the mesh by {\bf 135 degrees} anticlockwise around the Z axis, scaling up the mesh by {\bf 1.0925\%} on the Y axis, and translating it {\bf 6.82099667587} on the X axis and {\bf 8.7179839194} on the Y axis. Luckily, when doing this on the entire mesh is like doing it to its own points, which is the same of doing it to the singular waypoints, without needing to repeat the annotation process from the beginning: this was done with a Python script (numpy?, matrixes? "do, do, do" (?)).

\subsection*{Change JSON structure}

Then, since the JSON (?) was written as a list of dictionaries, it would be required to iterate over every element and search for the one with the searched name, and if they match, then it is possible to get the coordinates. Instead, I modified the file to be a dictionary, where each key represents the name of the room, and its respective value is a dictionary containing all the data related\footnote{related to that room}: room center and boundaries coordinates.

\subsection*{Remove a stub node from execution}

Currently, the only action supported is \code{ugv\_move}.
For testing purposes, some nodes were added to simulate \Acrshort{ugv}s and \Acrshort{uav}s: they were also started at the beginning of the execution, but one of these now has its real implementation, so its stub one has been removed from the execution. (?)

\section{Task executor}

Three nodes have been created to let the robot moves autonomously: 
\begin{itemize}
    \item a {\bf move client} that consists of two nodes executed in the same process, but in different threads, in order to listen and set the new destination of the robot
    \item a {\bf pose server} which is a service server node responsible for returning the current pose of the robot when requested
\end{itemize}

All the messages exchanged between nodes using services are part of \code{planning\_bridge\_msgs} package. % aggiungere immagini package? parlare anche di planning_bridge? 

\subsection{Move client} %(?)

As the name says, this process is used to move the robot around based on the planning decisions. It could be divided in two parts:

\begin{itemize}
    \item a planning client, which is always listening for the next goal to be reached
    \item a navigation client responsible for setting the destination and let the robot reach it
\end{itemize}

$$  
    PLANNING \;\; CLIENT \; 
        \autorightleftharpoons{float32 x, y, z}{bool success} \;
    NAVIGATION \;\; CLIENT
$$ 
    
% add it as a figure? add also planning server and navigation server? (?)

\subsubsection*{Planning client}

%inserire immagine codice (?)

It is an action client of \code{plansys2} (?), which means that after setting a goal it would send back feedback and the result of that action at the end. When a new task has been created and started, the planning server would call the \code{do\_work()} method of the designed client, to let it know a new job needs to be completed.
Inside this method the client transforms the room name in coordinates and passes the goal to the navigation client if it was not already set, making use of a custom service; after that, it asks its current position to the pose server (with another service) and uses it to set both the initial and current position; then, it returns a percentage of completion of the task. This method is called continuously until the task is completed, so if the initial position is already set (and as a result also the goal), it would continue to ask its current position in order to calculate the euclidean and return a new percentage of completion; because the path to reach the destination might not be a straight line and the distance used is the euclidean one, it could happen the current distance is greater than the initial one: in this case, they are swapped to keep 0 as the lower bound of completion. (?)

\subsubsection*{Navigation client}

This is both an action client and a service server: the first is used to receive the coordinates from the planning client and  uses the second one to connect to the navigation action server and set the given destination. It is only called the first time a new task is created.

\subsection{Pose server} %(?)
\label{sub:pose}

The pose server acts like a service server: when it receives a request, it returns the current position of the robot. In order to achieve it, some other calculations need to be done: we are interested in \code{base\_link} frame and the only transformation with it as a child frame is the \code{odom} one, but this is not the global one, the same used for waypoint coordinates. There is the need to express it from the \code{map} frame perspective, and luckily such a transformation is available thanks to \Acrshort{amcl} node. 
Thanks to \code{tf\_monitor}\cite{tfexample} used as a template, it was possible to obtain the correct transformation, fusing the required two.

$$
    PLANNING \;\; CLIENT \; 
        \autorightleftharpoons{\;\;\;\;\;\;\;\;\;\;\;\;\;\;\;\;\;\;\;\;\;\;\;}{float32 x, y, z} \;
    POSE \;\; SERVER
$$ 

% In order to have a better control over the task executions, it has been defined two services, in order to communicate and synchronize the robot action with the planning server. % continue





      \chapter{Future works}
\label{cha:futureworks}

Here it is a possible list of missing features that can be implemented in the future:

\begin{itemize}

\item The main part left is the full integration with drones as it was designed originally. This is due to the fact that operating with \acrshort{uav}s needs precise control in order to avoid collisions and before testing on a real drone, a quite expensive one, it is important to be sure that everything is working as expected.

\item Also, this system for ground robots (?) has been tested only with one \textit{shelfino} robot, because it is the only one currently accessible via SSH\footnote{Sicure SHell, a protocol that let a host connects to another one on the same network and using its terminal shell}, needed to set up it. Once this issue is resolved, a multi-robot test could be performed: with acrshort{Ros}2 using DDS (?), you can separate nodes running on a robot from those running on other robots. (?) % o "thanks to the fact that \acrshort{ros}2 uses DDS (?), it is possible to separate running nodes on a robot from the same ones running on other robots" come era originariament

\item Moreover, because of asynchronous development of each project, on the planning part was assumed that only \acrshort{uav}s are capable of taking photos, even if it could be done also with \acrshort{ugv}s: the implementation would be quite straightforward. From the specific actions designed for \acrshort{ugv}s, only \code{movement} and \code{move\_with\_uav}\footnote{the code is the same} ones were implemented, as opposed to \code{battery\_managment} and \code{charge}, because both do not have the needed infrastructure.

\item Another thing that it is possible to do, as soon it will be implemented on the planning system, is supporting live positioning: in this way also the user could have a real-time feedback in the web interface. Basically the only change consists on returning also the position of the robot (using the pose server described in \autoref{sub:pose}) alongside the current task completion percentage.

\item Talking about the navigation part, the obstacle avoidance could be enhanced, especially in the case of dynamic obstacles, because the robot sometimes takes some time to figure out what is going on and could end up colliding with them. A possible workaround could be increasing the minimum distance that should be kept between the robot and the unexpected obstacle, with some recovery action, like going back and replanning the path. A problem could also be the \code{ros1\_bridge} that adds some delay when exchanging messages between \acrshort{ros}1 and \acrshort{ros}2, ending up receiving lidar data not instantaneously.

\end{itemize}
      
    \endgroup

    % bibliografia in formato bibtex
    %
    % aggiunta del capitolo nell'indice
    \addcontentsline{toc}{chapter}{Bibliografia}
    % stile con ordinamento alfabetico in funzione degli autori
    \bibliographystyle{plain}
    \bibliography{biblio}

    \titleformat{\chapter}
        {\normalfont\Huge\bfseries}{Attachment \thechapter}{1em}{}
    % sezione Allegati - opzionale
    \appendix
    \input{attachments}
    
    \clearpage
    \addcontentsline{toc}{chapter}{Acronyms}
    \printnoidxglossaries

% \end{slashbreak}
\end{document}